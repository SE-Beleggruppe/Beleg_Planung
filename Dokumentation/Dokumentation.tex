%gibt an: Papierformat, einseitiger Druck, Schriftgr��e
\documentclass[a4paper,oneside,titlepage,12pt]{article}
%-------------------------------------------------------------------
\usepackage[a4paper, top=2cm, footskip=0pt, headheight=0.8cm, headsep=0.6cm, lmargin=3cm, rmargin=2cm]{geometry}
\usepackage{graphicx}
\usepackage{helvet}
\usepackage{amsmath}
\usepackage{amsthm}
\usepackage{amssymb}
\usepackage{hyperref} 
\usepackage[right]{eurosym}
\usepackage[latin1]{inputenc}

%--------------------------------------------------------------------
\renewcommand{\baselinestretch}{1.2}
%damit unten Platz f�r die Seitennummer bleibt

\begin{document}
%--------------------------------------------------------------------
%Titelseite
\begin{titlepage}
	\includegraphics{grafiken/HTW-Logo.png}
	%includegraphics[width=.3\textwidth]{grafiken/HTW-Logo.png}
	\vspace*{3cm}
	\begin{center}
		\Huge{Beleg - Dokumentation\\} \vspace*{1cm}
		\Large{
			Modul: Software Engineering II\\
			Entwicklung eines Software-Teilssystems zur Verwaltung
			von Beleggruppendaten\\}
		Chrisitan Knothe\\Martin Tzschoppe\\Benjamin Reim\\Markus
		Noack\\Benjamin Herzog\\Christian Schwarz\\Felix Krautschuk\\ 
		\vspace*{1cm}
		
		\vspace*{2cm}
		\normalsize{
			Studiengang Informatik\\
		}
	\end{center}
	\vspace{2cm}
\begin{center}
\large{Sommersemester 2014}
\end{center}
	\vspace*{3cm}



\end{titlepage}

\thispagestyle{empty}\clearpage

%-----------------------------------------------------------------------------------------------------------------------------
%\rmfamily \pagestyle{fancy} \setcounter{secnumdepth}{4}
\newtheorem{satz}{Satz}
\newtheorem{lemma}[satz]{Lemma}
\newtheorem{folgerung}[satz]{Folgerung}
\theoremstyle{definition}
\newtheorem{definition}[satz]{Definition}
\numberwithin{equation}{section}
\renewcommand{\proofname}{Beweis}

\pagenumbering{roman}\setcounter{page}{3} \tableofcontents
\newcounter{roemisch} \setcounter{roemisch}{\value{page}}
\clearpage
\section{Verantwortlichkeiten}
Projektleiter: Christian Knothe\\
Analyse: Martin Tzschoppe\\
Entwurf: Benjamin Reim\\
Datenbank: Markus Noack\\
Implementierung: Benjamin Herzog\\
Tests: Christian Schwarz\\
Dokumentation/Protokollierung: Felix Krautschuk
\section{Gruppen-Besprechungen}
\subsection{1. Treffen}
Termin: 07.04.2014\\
Tagesordung:
\begin{itemize}
  \item Verantwortlichkeiten festlegen
  \item Wahl des zu bearbeitenden Projekts
  \item Einigung auf Kommunikationsverfahren und -infrastrukturen
  \item Diskussionen �ber das Vorgehensmodell (agile Softwareentwicklung,
  Wasserfallmodell,\ldots)\\
\end{itemize}
Resultat:
\begin{itemize}
  \item Verantwortlichkeiten: siehe Auflistung
  \item Thema: Verwaltung von Beleggruppendaten (Alternative 5)
  \item Vorgehensweise: Wasserfallmodell
\end{itemize}

\newpage
\subsection{2. Treffen}
Termin: 23.04.2014 in der Bibliothek B302a\\
Tagesordnung:
\begin{itemize}
  \item erste Ideen f�r die Analyse der Problemstellung zusammentragen
  \item Klarheit �ber Komplexit�t und Detailliertheit der Features des Systems
  \item Entwurfsm�glichkeiten diskutieren
  verschaffen\\
\end{itemize}
Probleme die diskutiert wurden, aber noch mit Frau Hauptmann zu besprechen
sind:\\Wie sollen die Studenten-Gruppen �berhaupt verwaltet werden:
\begin{itemize}
  \item 1. M�glichkeit: Dozent tr�gt die Mitglieder der Gruppe selbst in das
  System ein
  \begin{itemize}
  	\item Dozent hat einen guten �berblick �ber alle Gruppen und sieht sofort,
    welche Gruppen bereits vollst�ndig sind und wo noch Mitglieder fehlen und
    welche Studenten noch keiner Gruppe zugeordnet wurden
    \item aber: dadurch hat der Dozent am meisten Aufwand bei der Organisation
    der Gruppen
  \end{itemize}
  \item 2. M�glichkeit: Dozent erstellt f�r die Gruppen eine Art Grundger�st
  (Rahmen) und die Studenten tragen dann in diese zun�chst leeren Gruppen ein
  \begin{itemize}
    \item  Dozent bleibt beim Eintragen der Gruppen passiv, hat nur
    Kontrollfunktion
    \item m�sste das Eintragen dann online (in einer Webanwendung) geschehen??
    \item w�re schwieriger zu implementieren als 1. Variante
  \end{itemize}
  \item 3. M�glichkeit: potentielle Gruppenleiter melden sich bei Dozenten und
  werden eingetragen
  \begin{itemize}
  \item diese suchen sich ihre Mitglieder selbst
  \item es gibt somit keine leeren Gruppen
  \item Organisation der Gruppen geschieht ebenfalls durch Studenten selbst
  \end{itemize}
\end{itemize}
Filterm�glichkeiten??\\
wie komplex soll das System insgesamt werden?? Viele Features oder einfach
halten??\\
Was umfasst der Entwurf alles??\\
----------------------\\
Plan f�r n�chstes Treffen:\\
Termin: 30.04.2014\\
Benjamin Reim: Dialoge, eventuell erste Entw�rfe\\
Markus Noack: Archivierung, alles was zum Schluss aus der Datenbank in der
PDF-Datei stehen muss
\setcounter{page}{2} \pagenumbering{arabic}
\end{document}
