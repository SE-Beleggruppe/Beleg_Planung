%gibt an: Papierformat, einseitiger Druck, Schriftgr��e
\documentclass[a4paper,oneside,titlepage,12pt]{article}
%-------------------------------------------------------------------
\usepackage[a4paper, top=2cm, footskip=0pt, headheight=0.8cm, headsep=0.6cm, lmargin=3cm, rmargin=2cm]{geometry}
\usepackage{graphicx}
\usepackage{helvet}
\usepackage{amsmath}
\usepackage{amsthm}
\usepackage{amssymb}
\usepackage{hyperref} 
\usepackage[right]{eurosym}
\usepackage[latin1]{inputenc}

%--------------------------------------------------------------------
\renewcommand{\baselinestretch}{1.2}

\begin{document}
%--------------------------------------------------------------------
%Titelseite
\begin{titlepage}
	\includegraphics{grafiken/HTW-Logo.png}
	%\includegraphics[width=.3\textwidth]{grafiken/HTW-Logo.png}
	\vspace*{3cm}
	\begin{center}
		\Huge{Entwicklerdokumentation\\} \vspace*{1cm}
		\huge{Case-Gruppe 04\\}
		\vspace*{1cm}
		\Large{
			Modul: Software Engineering II\\}
		\vspace*{2cm}
		\normalsize{
			Studiengang Informatik\\
		}
	\end{center}
	\vspace{2cm}
\begin{center}
\large{Sommersemester 2014}
\end{center}
	\vspace*{3cm}



\end{titlepage}

\thispagestyle{empty}\clearpage

%-----------------------------------------------------------------------------------------------------------------------------------
%\rmfamily \pagestyle{fancy} \setcounter{secnumdepth}{4}
\newtheorem{satz}{Satz}
\newtheorem{lemma}[satz]{Lemma}
\newtheorem{folgerung}[satz]{Folgerung}
\theoremstyle{definition}
\newtheorem{definition}[satz]{Definition}
\numberwithin{equation}{section}
\renewcommand{\proofname}{Beweis}

\pagenumbering{roman}\setcounter{page}{3} \tableofcontents
\newcounter{roemisch} \setcounter{roemisch}{\value{page}}
\clearpage

\setcounter{page}{2} \pagenumbering{arabic}


\section{Einleitung}
Die vorliegende Dokumentation dient zuk�nftigen Entwicklern dazu sich in das
bestehende System einarbeiten zu k�nnen und die Arbeit fortsetzen zu k�nnen.

\section{Enwurf}
\subsection{Grobentwurf}
\subsubsection{Architektur}
\subsubsection{Paketdiagramm}


\section{Implementierungsentwurf}
Unser Programm ist in insgesamt 4 Pakete gegliedert. Die beiden Pakete
DozentBelegverwaltungUI und StudentBelegverwaltungUI enthalten s�mtliche Klassen
f�r die Benutzeroberfl�chen (Login, Bearbeitung, \ldots) des Dozenten- und des
Studentenprogrammes. Sie verwenden die Klassen aus dem Typen-paket und greifen
gemeinsam auf die Datenbank des Paketes DB\_Services zu.\\
Im folgenden werden f�r jedes Paket die Klassen aus dem Grobentwurf mit ihren
Methoden definiert und ihre Aufgaben beschrieben.

\subsection{Paket Typen}
\subsubsection{Beleg}
.
\subsubsection{Thema}
.
\subsubsection{Gruppe}
.
\subsubsection{Rolle}
.
\subsubsection{Dozent}
.
\subsubsection{Student}
.


\subsection{Paket DozentBelegverwaltungUI}
\subsubsection{MainForm}
\subsubsection{BelegBearbeiten}
\subsubsection{GruppeBearbeiten}
\subsubsection{RolleVerwalten}
\subsubsection{ThemenVerwalten}
\subsubsection{PdfArchivierung}
\subsubsection{KontaktForm}
\subsubsection{Eingabe}


\subsection{Paket StudentBelegverwaltungUI}
\subsubsection{Die Klasse LoginForm}
\subsubsection{Die Klasse MainForm}
\subsubsection{Die Klasse FormErstanmeldung}
\subsubsection{Die Klasse FormMitgliederNeuEingeben}


\subsection{Paket DBService}
\subsubsection{Die Klasse Database}


\end{document}