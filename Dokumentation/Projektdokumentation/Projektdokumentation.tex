%gibt an: Papierformat, einseitiger Druck, Schriftgr��e
\documentclass[a4paper,oneside,titlepage,12pt]{article}
%-------------------------------------------------------------------
\usepackage[a4paper, top=2cm, footskip=0pt, headheight=0.8cm, headsep=0.6cm, lmargin=3cm, rmargin=2cm]{geometry}
\usepackage{graphicx}
\usepackage{helvet}
\usepackage{amsmath}
\usepackage{amsthm}
\usepackage{amssymb}
\usepackage{hyperref} 
\usepackage[right]{eurosym}
\usepackage[latin1]{inputenc}

%--------------------------------------------------------------------
\renewcommand{\baselinestretch}{1.2}

\begin{document}
%--------------------------------------------------------------------
%Titelseite
\begin{titlepage}
	\includegraphics{grafiken/HTW-Logo.png}
	%\includegraphics[width=.3\textwidth]{grafiken/HTW-Logo.png}
	\vspace*{3cm}
	\begin{center}
		\Huge{Projektdokumentation\\} \vspace*{1cm}
		\huge{Case-Gruppe 04\\}
		\vspace*{1cm}
		\Large{
			Modul: Software Engineering II\\}
		\vspace*{2cm}
		\normalsize{
			Studiengang Informatik\\
		}
	\end{center}
	\vspace{2cm}
\begin{center}
\large{Sommersemester 2014}
\end{center}
	\vspace*{3cm}



\end{titlepage}

\thispagestyle{empty}\clearpage

%-----------------------------------------------------------------------------------------------------------------------------------
%\rmfamily \pagestyle{fancy} \setcounter{secnumdepth}{4}
\newtheorem{satz}{Satz}
\newtheorem{lemma}[satz]{Lemma}
\newtheorem{folgerung}[satz]{Folgerung}
\theoremstyle{definition}
\newtheorem{definition}[satz]{Definition}
\numberwithin{equation}{section}
\renewcommand{\proofname}{Beweis}

\pagenumbering{roman}\setcounter{page}{3} \tableofcontents
\newcounter{roemisch} \setcounter{roemisch}{\value{page}}
\clearpage

\setcounter{page}{2} \pagenumbering{arabic}


\section{Motivation}
Das Ziel bei diesem Projekt war es, unser Fachwissen und die erlernten
Techniken aus dem letzten Semester und aus diesem Semester anzuwenden und bei
der Planung, dem Entwurf, der Implementierung und dem Test unserer Anwendung die
Prozesse der Softwareentwicklung kennenzulernen.


\section{Ablauf des Projektes}
\subsection{Meilensteine}
\begin{itemize}
  \item 1. Abgabe des Pflichtenheftes
  \item 2. Abschluss des Feinentwurfes
  \item 3. Abschluss der Implementierung
  \item 4. Test und gegebenenfalls Durchf�hrung notwendiger �nderungen
  \item 5. Abgabe des entwickelten Systems an den Kunden mit Pr�sentation
\end{itemize}

\subsection{Vorgehensweise}
\subsubsection{Gruppenfindung und Themenwahl}
Da sich die Gruppenmitglieder bereits w�hrend des Studiums kennenlernten und
viele Pr�fungen gemeinsam vorbereiteten, ging der Prozess der Gruppenfindung
sehr schnell. \\Auch die Einigung auf das zu bearbeitende Thema nahm kaum Zeit
in Anspruch. Zu unserer ersten Gruppensitzung (07.04.2014) entschieden wir uns
f�r die Entwicklung eines Systems zur Verwaltung von Beleggruppendaten (Alternative 5), da uns diese Aufgabe am
sinnvollsten erschien. Au�erdem legten wir folgende Rollenverteilung fest:
\begin{itemize}
  \item Projektleiter: Christian Knothe
  \item Analyse: Martin Tzschoppe
  \item Entwurf: Benjamin Reim
  \item Datenbank: Markus Noack
  \item Implementierung: Benjamin Herzog
  \item Tests: Christian Schwarz
  \item Dokumentation/Protokollierung: Felix Krautschuk
\end{itemize}
Unser Projektleiter Christian Knothe legte mehr Wert auf Eigeninitiative der
Gruppenmitglieder als auf seine f�hrende Rolle und beschr�nkte seine
Befugnisse in Entscheidungen auf das N�tigste, was von uns Gruppenmitgliedern
sehr begr��t wurde.
Die Aufgabenverteilung geschah generell durch die Verantwortlichen der
jeweiligen Bereiche. Somit konnte jeder sein eigenes F�hrungspotenzial in seinem
zu Beginn zugeteilten Gebiet unter Beweis stellen, die Verantwortlichen standen
in ihrem Gebiet durch die Aufgabenverteilung nie allein da und so konnte jedes
Gruppenmitglied Erfahrung in allen Aufgabenbereichen sammeln.

\subsubsection{Gruppensitzungen}
Zu unserem ersten Treffen am 07.04.2014 legten wir fest dass ein Treffen aller
Gruppenmitglieder alle 14 Tage Montags (ungerade Woche) ausreichend sei. Da es
aber besonders zu Beginn des Projektes sehr viele Unklarheiten und Diskussionen
bez�glich des Aufbaus und des Umfanges des gew�nschten Softwaresystems gab,
�nderten wir den Rhythmus und trafen uns �fter. Unser Projektleiter schrieb
daf�r rechtzeitig eine Email mit einer Tagesordnung an alle Mitglieder der
Gruppe, sowie an unsere Kundin Frau Prof. Dr. Hauptmann.


\subsubsection{Fertigstellung und Abgabe des Pflichtenheftes}
Da die Analyse der Ausgangspunkt f�r die Entstehung des Softwaresystems
ist, stellte es sich f�r uns als sinnvoll heraus wenn sich das gesamte
Team mit der Analyse besch�ftigt damit wir schnell mit dem Pflichtenheft
vorankommen. Allerdings entstanden so viele Diskussionen die unseren Fortschritt
bei der Bearbeitung der Aufgabe behinderten. Aufgrund mehrerer ungl�cklcher
Versuche ein gemeinsames Treffen mit Frau Hauptmann zu vereinbaren, um mit ihr
solche Probleme zu beseitigen, konnten wir erst sehr sp�t mit der eigentlichen
Erstellung von Anforderungen, Kontextdiagrammen usw beginnen. Nach Absprache mit
Frau Hauptmann erfolgte schlie�lich die Abgabe des Pflichtenheftes eine Woche
nach dem offiziellen Abgabetermin.

\subsubsection{Entwicklungsumgebung und Programmiersprache}
Obwohl Frau Hauptmann keinen Wert auf Plattformunabh�ngigkeit legte und die
Vorgabe nur lautete, dass das System unter Windows laufen sollte, nahmen wir uns
vor ein plattformunabh�ngiges Produkt in Java oder in C++ mit QT zu entwickeln. 
Beim Entwickeln eines Oberfl�chenprototypen entstanden bei der Nutzung der eben
genannten Systeme Probleme, weshalb wir uns schlie�lich f�r die Programmierung
mit CSharp unter Visual Studio entschieden.


\section{Kommunikation}
\subsection{Gruppensitzungen}
Zu unseren Konferenzen nahmen immer alle Mitglieder Teil, da f�r uns jede
Meinung bei den Diskussionen wichtig war. So dauerten die Diskussionen zwar oft
etwas l�nger, doch sank dadurch das Risiko, dass etwas unbedacht blieb und erst
erst zu sp�t besprochen wurde. Au�erdem hatte die Anwesenheit aller
Teammitglieder den Vorteil, dass jeder auf dem neusten Stand bez�glich des
Projektes war. So konnte man sich das Informieren fehlender Gruppenmitglieder
�ber die neusten Erkenntnisse sparen.\\
Zu Beginn jeder Sitzung wurden die Zwischenergebnisse der vergangenen Tage seit
dem letzten Treffen von den jeweiligen Gruppenmitgliedern vorgestellt. Diese
wurden dann von der gesamten Gruppe diskutiert.\\
Zum Ende einer Sitzung wurden die bis zum n�chsten Treffen zu erledigenden
Aufgaben verteilt und die Tagesordnung des n�chsten Treffens festgelegt.
Leider kam es relativ oft vor, dass einzelne Punkte der Tagesordnung einer
Sitzung auf die Tagesordnung der n�chsten Konferenz verschoben werden mussten.

\subsection{Forum}
\subsection{Github}
\subsection{Facebook}

\end{document}
